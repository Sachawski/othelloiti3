\begin{algorithme}
  \small
  \fonction
  {pionMemeCouleur}
  {p: Plateau, coup: Coup, pionLegal: Coups}
  {Coups}
  {
    pionMCouleur: Coups\\
    nb, i: \naturel\\
    couptemp: coup\\
    x, y: \naturel\\
    directionX,directionY: \naturel\\
    recherche: \booleen
    pos: Position\\
  }
  {
    \affecter{pionMCouleur}{coups()}
    \affecter{nb}{nbCoups(pionLegal)}
    \pour{i}{1}{nb}{}
    {
      \affecter{recherche}{FAUX}
      \affecter{couptemp}{iemeCoup(pionLegal, i)}
      \affecter{x}{obtenirX(obtenirPositionCoup(couptemp))}
      \affecter{y}{obtenirY(obtenirPositionCoup(couptemp))}
      \affecter{directionX}{x-obtenirX(coup)}
      \affecter{directionY}{y-obtenirY(coup)}
      \tantque{non(recherche)}
      {
        \affecter{x}{x+ directionX}
        \affecter{y}{y+ directionY}
        \sialorssinon{x$\geq$ 1 et x$\leq$ 8 et y$\geq$ 1 et y$\leq$ 8}
        {
          \affecter{pos}{position(x, y)}
          \sialors{estCaseVide(p, pos)}{
            \affecter{recherche}{1}
          }
          \sialors{obtenirCouleurSuperieure(obtenirPion(pos, p))= obtenirCouleurSuperieure(coupObtenirPionCoup(coup))}
          {
            ajouterCoups(pionMCouleur,coupCoup(obtenirPion(pos, p), pos))\\
            \affecter{recherche}{1}
          }
        }
        {
          \affecter{recherche}{1}
        }
      }
    }
    \retourner{pionMCouleur}
  }
  \end{algorithme}
  