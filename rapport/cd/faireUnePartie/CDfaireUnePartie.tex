\begin{algorithme}
  \small
  \fonction
      {initialiserPlateau}
      {}
      {Plateau}
      {p:Plateau\\
        pPion:Pion\\
        i,j:\naturel\\
        pos:Position}
      { 
        \affecter{p}{plateau()}
        \pour{i}{4}{5}{}
             {
               \pour{j}{4}{5}{}
                    {
                      \sialorssinon{i=j}
                                   {
                                     \affecter{pPion}{creerPion(blanc())}
                                     \affecter{pos}{position(i,j)}\\
                                     \instruction{poserPion(pos,pPion,p)}
                                   }
                                   {
                                     \affecter{pPion}{creerPion(noir())}
                                     \affecter{pos}{position(i,j)}\\
                                     \instruction{poserPion(pos,pPion,p)}
                                   }
                    }
             }
             \retourner{p}
      }
\end{algorithme}

\vspace{10mm}

\begin{algorithme}
  \small
  \fonction
      {adversairesAdjacent}
      {p: Plateau, coup: Coup}
      {Coups}
      {
        recherche: \booleen\\
        coupslegal: Coups\\
        ldelta, cdelta: \entier\\
        minx, miny, maxx, maxy: \entier\\
        pos,postemp: Position\\
        x, y: entier\\
      }
      {
        \affecter{recherche}{VRAI}
        \affecter{coupslegal}{coups()}
        \affecter{pos}{coupObtenirPositionCoup(coup)}
        \affecter{minx}{max(1, obtenirX(pos)- 1)}
        \affecter{miny}{max(1, obtenirY(pos)- 1)}
        \affecter{maxx}{min(8, obtenirX(pos)+ 1)}
        \affecter{maxy}{min(8, obtenirY(pos)+ 1)}
        \affecter{y}{miny}
        \tantque{y$\leq$ maxy et recherche}
                {
                  \affecter{x}{minx}
                  \tantque{x$\leq$ maxx et recherche}
                          {
                            \affecter{postemp}{postion(x,y)}
                            \sialors{non(estCaseVide(p,postemp) et postemp$\neq$ pos)}
                                    {
                                      \sialors{obtenirCouleurSuperieur(obtenirPion(p,postemp))$\neq$ obtenirCouleurSuperieur(obtenirPionCoup(coup))}
                                              {
                                                \instruction{ajouterCoups(coupslegal,obtenirPion(p,postemp))}
                                              }
                                    }
                                    \affecter{x}{x+1}
                          }
                          \affecter{y}{y+1}
                }
                \retourner{coupslegal}
      }
\end{algorithme}

\vspace{10mm}

\begin{algorithme}
  \small
  \fonction
      {pionMemeCouleur}
      {p: Plateau, coup: Coup, pionLegal: Coups}
      {Coups}
      {
        pionMCouleur: Coups\\
        nb, i: \naturel\\
        couptemp: coup\\
        x, y: \naturel\\
        directionX,directionY: \naturel\\
        recherche: \booleen
        pos: Position\\
      }
      {
        \affecter{pionMCouleur}{coups()}
        \affecter{nb}{nbCoups(pionLegal)}
        \pour{i}{1}{nb}{}
             {
               \affecter{recherche}{FAUX}
               \affecter{couptemp}{iemeCoup(pionLegal, i)}
               \affecter{x}{obtenirX(obtenirPositionCoup(couptemp))}
               \affecter{y}{obtenirY(obtenirPositionCoup(couptemp))}
               \affecter{directionX}{x-obtenirX(coup)}
               \affecter{directionY}{y-obtenirY(coup)}
               \tantque{non(recherche)}
                       {
                         \affecter{x}{x+ directionX}
                         \affecter{y}{y+ directionY}
                         \sialorssinon{x$\geq$ 1 et x$\leq$ 8 et y$\geq$ 1 et y$\leq$ 8}
                                      {
                                        \affecter{pos}{position(x, y)}
                                        \sialors{estCaseVide(p, pos)}
                                                {
                                                  \affecter{recherche}{1}
                                                }
                                        \sialors{obtenirCouleurSuperieure(obtenirPion(pos, p))$\neq$= obtenirCouleurSuperieure(coupObtenirPionCoup(coup))}
                                                {
                                                  \instruction{ajouterCoups(pionMCouleur,coupCoup(obtenirPion(pos, p), pos))}\\
                                                  \affecter{recherche}{1}
                                                }
                                      }
                                      {
                                        \affecter{recherche}{1}
                                      }
                       }
             }
             \retourner{pionMCouleur}
      }
\end{algorithme}

\vspace{10mm}
\begin{algorithme}
  \small
  \fonction
      {coupLegal}
      {p: Plateau,coup: Coup}
      {\booleen}
      {
        pos: Position\\
        pionLegal: Coups\\
      }
      {

        \affecter{pos}{coupObtenirPositionCoup(coup)}
        \sialorssinon{caseVide(p, pos)}
                     {
                       \sialorssinon{nbCoups(adversairesAdjacent(p, coup))$\neq$ 0}
                                    {
                                      \affecter{pionLegal}{adversairesAdjacent(p, coup)}
                                      \sialorssinon{nbCoups(pionMêmeCouleur(p, coup, pionLegal))$\neq$ 0}
                                                   {
                                                     \retourner{VRAI}
                                                   }
                                                   {
                                                     \retourner{FAUX}
                                                   }
                                    }
                                    {
                                      \retourner{FAUX}
                                    }
                     }
                     {
                       \retourner{FAUX}
                     }
      }
\end{algorithme}

\vspace{10mm}

\begin{algorithme}
  \small
  \procedure{etatPartie}{\paramEntree{plateau:Plateau}, \paramEntreeSortie{couleur: Couleur}, \paramEntreeSortie{etat: EtatPartie}}
            {
              scoreBlanc: \entier
              scoreNoir: \entier
            }
            {
              \sialors{non(plateauBloque(plateau))}{
                \affecter{etat}{partieEnCours}
              }
              \sialors{plateauBloque(plateau)}{
                \affecter{scoreBlanc}{evaluerNb(plateau, BLANC)}
                \affecter{scoreNoir}{evaluerNb(plateau, NOIR)}
                \sialorssinon{scoreBlanc$>$scoreNoir}{
                  \affecter{etat}{partieGagnee}
                  \affecter{couleur}{BLANC}
                }
                             {
                               \sialorssinon{scoreNoir$>$scoreBlanc}{
                                 \affecter{etat}{partieGagnee}
                                 \affecter{couleur}{NOIR}
                               }
                                            {
                                              \affecter{etat}{partieEgal}
                                            }
                             }
              }
            }
\end{algorithme}

\vspace{10mm}

\begin{algorithme}
  \procedure{retournerPionEmprisonnes}{\paramEntreeSortie{plateau : Plateau} , \paramEntree{coup : Coup}}
            {coupTemp : Coup , x, y, xCoup, yCoup ,directionX, directionY, i :\naturel, coupsEmprisonnants : Coups, pos : Position}
            {
              \affecter{xCoup}{obtenirX(obtenirPositionCoup(coup))}
              \affecter{yCoup}{obtenirY(obtenirPositionCoup(coup))}
              \affecter{coupsEmprisonnants}{pionMemeCouleur(plateau,coup,coupsAdjacentsAdversaires(plateau,coup))}
              \pour{i}{1}{nbCoups(coupsEmprisonnants)}{}{
                \affecter{coupTemp}{iemeCoup(coupsEmprisonnant,i)}
                \affecter{x}{obtenirX(obtenirPositionCoup(coupTemp))}
                \affecter{y}{obtenirY(obtenirPositionCoup(coupTemp))}
                \sialorssinon{x-xCoup=0}{\affecter{directionX}{0}
                  \affecter{directionY}{$\frac{y-obtenirY(coup)}{abs(y-obtenirY(coup))}$}}
                             {\sialorssinon{y-yCoup=0}{\affecter{directionY}{0}
                                 \affecter{directionX}{$\frac{x-xCoup}{abs(x-xCoup)}$}}{
                                 \affecter{directionX}{$\frac{x-xCoup}{abs(x-xCoup)}$}
                                 \affecter{directionY}{$\frac{y-yCoup}{abs(y-yCoup}$}
                               }
                             }
                             \affecter{x}{x+directionX}
                             \affecter{y}{y+directionY}
                             \tantque{x$\neq$xCoup || y$\neq$yCoup}
                                     {
	                               \affecter{pos}{position(x,y)}
	                               \instruction{Plateau.retournerPion(plateau,pos)}
	                               \affecter{x}{x+directionX}
	                               \affecter{y}{y+directionY}
	                               
	                               
                                     }
              }
            }
\end{algorithme}

\vspace{10mm}

\begin{algorithme}
  \procedureAvecPreconditions
      {jouer}
      {\paramEntreeSortie{plateau : Plateau} , \paramEntree{coup : Coup}}
      {coupLegal(coup)}
      {}
      {
        \instruction{retournerPionsEmprisonnes(plateau,coup)}
        \instruction{Plateau.poserPion(plateau,CP.position(coup), CP.pion(coup))} 
      }
\end{algorithme}
                             

\begin{algorithme}
  \small
  \procedure
  {faireUnePartie}
  {\paramEntree{obtenirCoupBlanc,obtenirCoupJoueurNoir : ObtenirCoupEnFctDuJoueur}, \paramEntree{sortie : Sortie} \paramEntreeSortie{couleurGagnant: Couleur}, \paramEntreeSortie{etat : EtatPartie, couleurGagnant : Couleur}}
  {
    plateau: Plateau
    joueurCourant: Pion
    prochainCoup: Coup
    partieEnCours: etat
    start,end: timeT
  }
  { 
    \affecter{joueurCourant}{creerPion(BLANC)}
    \affecter{plateau}{initialiserPlateau()}
    \repeter{
      \affecter{start}{time(NULL)}
      \repeter{
        \affecter{prochainCoup}{coupEnFctJoueur(obtenirCoupBlanc, obtenirCoupNoir, obtenirCouleurSuperieure(joueurCourant), plateau)}
      }{(coupLegal(plateau,prochainCoup))}
              \instruction{jouer(plateau, prochainCoup)}
              \affecter{end}{time(NULL)}
              \sialors{(end-start)$<$1}
                      {
                        \instruction{attendre(1)}
                      }
              \instruction{sortie(plateau,prochainCoup,1)}
              \instruction{retournerPion(joueurCourant)}
                          {
                            \sialors{plateauBloquePourUneCouleur(plateau,obtenirCouleurSuperieure(joueurCourant))}
                            \instruction{retournerPion(joueurCourant)}
                            \instruction{sortie(plateau,prochainCoup,0)}
                          }
                          \instruction{etatPartie(plateau,couleurGagnant,etat)}
    }{non(etat = partieEnCours)}
  }

\end{algorithme}

\vspace{10mm}

\begin{algorithme}
  \small
  \fonction
      {coupEnFctJoueur}
      {obtenirCoupBlanc,obtenirCoupNoir : ObtenirCoupEnFctDuJoueur,couleur : Couleur, plateau : Plateau}
      {Coup}
      {coup : Coup}
      {
        \sialorssinon{couleur = NOIR}
                {
                  \affecter{coup}{obtenirCoupNoir(plateau, NOIR, PROFONDEUR)}
                }
                {
                  \affecter{coup}{obtenirCoupBlanc(plateau, BLANC, PROFONDEUR)}
                }
                \retourner{coup}
      }
\end{algorithme}
