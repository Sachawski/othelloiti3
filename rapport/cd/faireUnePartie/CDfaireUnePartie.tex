\begin{algorithme}
  \small
  \fonction
      {initialiserPlateau}
      {}
      {\textbf{Plateau}}
      {plateau : \textbf{Plateau}, pion : \textbf{Pion}, i,j : \naturel, pos : \textbf{Position}}
      { 
        \affecter{plateau}{plateau()}

        poserPion(plateau,Position(3,3),creerPion(BLANC))\\
        poserPion(plateau,Position(3,4),creerPion(NOIR))\\
        poserPion(plateau,Position(4,3),creerPion(NOIR))\\
        poserPion(plateau,Position(4,4),creerPion(BLANC))\\
        \retourner{plateau}
      }
\end{algorithme}

\vspace{5mm}

\begin{algorithme}
  \small
  \fonction
      {adversairesAdjacents}
      {plateau: \textbf{Plateau}, coup: \textbf{Coup}}
      {\textbf{Coups}}
      {
        recherche: \booleen\\
        coupsLegal: \textbf{Coups}\\
        ldelta, cdelta: \entier\\
        minx, miny, maxx, maxy, x, y: \entier\\
        pos,postemp: \textbf{Position}\\
      }
      {
        \affecter{coupsLegal}{coups()}
        \affecter{pos}{coupObtenirPositionCoup(coup)}
        \affecter{minx}{max(1, obtenirX(pos)- 1)}
        \affecter{miny}{max(1, obtenirY(pos)- 1)}
        \affecter{maxx}{min(8, obtenirX(pos)+ 1)}
        \affecter{maxy}{min(8, obtenirY(pos)+ 1)}
        \affecter{y}{miny}
        \tantque{y$\leq$ maxy}
                {
                  \affecter{x}{minx}
                  \tantque{x$\leq$ maxx}
                          {
                            \affecter{postemp}{postion(x,y)}
                            \sialors{non(estCaseVide(plateau,postemp) et postemp $\neq$ pos)}
                                    {
                                      \sialors{obtenirCouleurSuperieur(obtenirPion(plateau,postemp)) $\neq$ obtenirCouleurSuperieur(obtenirPionCoup(coup))}
                                              {
                                                \instruction{ajouterCoups(coupsLegal,obtenirPion(plateau,postemp))}
                                              }
                                    }
                                    \affecter{x}{x+1}
                          }
                          \affecter{y}{y+1}
                }
                \retourner{coupsLegal}
      }
\end{algorithme}

\vspace{5mm}

\begin{algorithme}
  \small
  \fonction
      {pionMemeCouleur}
      {plateau : \textbf{Plateau}, coup : \textbf{Coup}, pionLegaux : \textbf{Coups}}
      {Coups}
      {
        pionMCouleur: \textbf{Coups}\\
        nb, i: \naturel\\
        couptemp: \textbf{Coup}\\
        x, y: \naturel\\
        directionX,directionY: \naturel\\
        recherche: \booleen
        pos : \textbf{Position} \\
      }
      {
        \affecter{pionMCouleur}{coups()}
        \affecter{nb}{nbCoups(pionLegaux)}
        \pour{i}{1}{nb}{}
             {
               \affecter{recherche}{FAUX}
               \affecter{couptemp}{iemeCoup(pionLegaux, i)}
               \affecter{x}{obtenirX(obtenirPositionCoup(couptemp))}
               \affecter{y}{obtenirY(obtenirPositionCoup(couptemp))}
               \affecter{directionX}{x-obtenirX(coup)}
               \affecter{directionY}{y-obtenirY(coup)}
               \tantque{non(recherche)}
                       {
                         \affecter{x}{x+ directionX}
                         \affecter{y}{y+ directionY}
                         \sialorssinon{x$\geq$ 1 et x$\leq$ 8 et y$\geq$ 1 et y$\leq$ 8}
                                      {
                                        \affecter{pos}{position(x, y)}
                                        \sialorssinon{estCaseVide(plateau, pos)}
                                                {
                                                  \affecter{recherche}{VRAI}
                                                }
                                                {
                                                                                        			\sialors{obtenirCouleurSuperieure(obtenirPion(pos, plateau)) = obtenirCouleurSuperieure(coupObtenirPionCoup(coup))}
                                                {
                                                  \instruction{ajouterCoups(pionMCouleur,coupCoup(obtenirPion(pos, plateau), pos))}\\
                                                  \affecter{recherche}{VRAI}
                                                }
                                                
                                                }

                                      }
                                      {
                                        \affecter{recherche}{VRAI}
                                      }
                       }
             }
             \retourner{pionMCouleur}
      }
\end{algorithme}

\vspace{5mm}

\begin{algorithme}
  \small
  \fonction
      {coupLegal}
      {plateau : \textbf{Plateau}, coup: \textbf{Coup}}
      {\booleen}
      {
        pos: \textbf{Position}\\
        pionLegaux: \textbf{Coups}\\
      }
      {

        \affecter{pos}{coupObtenirPositionCoup(coup)}
        \sialorssinon{caseVide(plateau, pos)}
                     {
                       \sialorssinon{nbCoups(adversairesAdjacents(plateau, coup))$\neq$ 0}
                                    {
                                      \affecter{pionLegaux}{adversairesAdjacents(plateau, coup)}
                                      \sialorssinon{nbCoups(pionMemeCouleur(plateau, coup, pionLegaux))$\neq$ 0}
                                                   {
                                                     \retourner{VRAI}
                                                   }
                                                   {
                                                     \retourner{FAUX}
                                                   }
                                    }
                                    {
                                      \retourner{FAUX}
                                    }
                     }
                     {
                       \retourner{FAUX}
                     }
      }
\end{algorithme}

\vspace{5mm}

\begin{algorithme}
  \small
  \procedure{etatPartie}{\paramEntree{plateau : \textbf{Plateau}}, \paramSortie{couleur: \textbf{Couleur}, etat: \textbf{EtatPartie}}}
            {
              scoreBlanc: \entier,
              scoreNoir: \entier
            }
            {
              \sialors{non(plateauBloque(plateau))}{
                \affecter{etat}{partieEnCours}
              }
              \sialors{plateauBloque(plateau)}{
                \affecter{scoreBlanc}{evaluerNb(plateau, BLANC)}
                \affecter{scoreNoir}{evaluerNb(plateau, NOIR)}
                \sialorssinon{scoreBlanc$>$scoreNoir}{
                  \affecter{etat}{partieGagnee}
                  \affecter{couleur}{BLANC}
                }
                             {
                               \sialorssinon{scoreNoir$>$scoreBlanc}{
                                 \affecter{etat}{partieGagnee}
                                 \affecter{couleur}{NOIR}
                               }
                                            {
                                              \affecter{etat}{partieEgal}
                                            }
                             }
              }
            }
\end{algorithme}

\vspace{5mm}

\begin{algorithme}
  \small
  \procedure
      {retournerPionsEmprisonnes}
      {\paramEntreeSortie{plateau : \textbf{Plateau}} , \paramEntree{coup : \textbf{Coup}}}
      {coupTemp : Coup , x, y, xCoup, yCoup ,directionX, directionY, i :\naturel, coupsEmprisonnants : Coups, pos : \textbf{Position}}
      {
        \affecter{xCoup}{obtenirX(obtenirPositionCoup(coup))}
        \affecter{yCoup}{obtenirY(obtenirPositionCoup(coup))}
        \affecter{coupsEmprisonnants}{pionMemeCouleur(plateau,coup,adversairesAdjacents(plateau,coup))}
        \pour{i}{1}{nbCoups(coupsEmprisonnants)}{}{
          \affecter{coupTemp}{iemeCoup(coupsEmprisonnants,i)}
          \affecter{x}{obtenirX(obtenirPositionCoup(coupTemp))}
          \affecter{y}{obtenirY(obtenirPositionCoup(coupTemp))}
          \sialorssinon{x-xCoup=0}
                       {
                         \affecter{directionX}{0}
                         \affecter{directionY}{$\frac{y-obtenirY(coup)}{abs(y-obtenirY(coup))}$}
                       }
                       {
                         \sialorssinon{y-yCoup=0}
                                      {
                                        \affecter{directionY}{0}
                                        \affecter{directionX}{$\frac{x-xCoup}{abs(x-xCoup)}$}
                                      }
                                      {
                                        \affecter{directionX}{$\frac{x-xCoup}{abs(x-xCoup)}$}
                                        \affecter{directionY}{$\frac{y-yCoup}{abs(y-yCoup}$}
                                      }
                       }
                       \affecter{x}{x+directionX}
                       \affecter{y}{y+directionY}
                       \tantque{x$\neq$xCoup || y$\neq$yCoup}
                               {
	                         \affecter{pos}{position(x,y)}
	                         \instruction{retournerPion(plateau,pos)}
	                         \affecter{x}{x+directionX}
	                         \affecter{y}{y+directionY}
                               }
        }
      }
\end{algorithme}

\vspace{5mm}

\begin{algorithme}
  \small
  \procedureAvecPreconditions
      {jouer}
      {\paramEntreeSortie{plateau : \textbf{Plateau}} , \paramEntree{coup : \textbf{Coup}}}
      {coupLegal(coup)}
      {}
      {
        \instruction{retournerPionsEmprisonnes(plateau,coup)}
        \instruction{poserPion(plateau,position(coup), pion(coup))} 
      }
\end{algorithme}
                             
  \vspace*{5mm}

\begin{algorithme}
  \small
  \procedure
  {faireUnePartie}
  {\paramEntree{obtenirCoupBlanc,obtenirCoupNoir : \textbf{ObtenirCoupEnFctDuJoueur}, sortie : \textbf{Sortie}} \paramSortie{couleurGagnant: \textbf{Couleur}, etat : \textbf{EtatPartie}}}
  {
    plateau: \textbf{Plateau}
    joueurCourant: \textbf{Pion}
    prochainCoup: \textbf{Coup}
    partieEnCours: \textbf{EtatPartie}
  }
  { 
    \affecter{joueurCourant}{creerPion(BLANC)}
    \affecter{plateau}{initialiserPlateau()}
    \repeter{
      \repeter{
        \affecter{prochainCoup}{coupEnFctJoueur(obtenirCoupBlanc, obtenirCoupNoir, obtenirCouleurSuperieure(joueurCourant), plateau)}
      }{(coupLegal(plateau,prochainCoup))}
      \instruction{jouer(plateau, prochainCoup)}
      \instruction{sortie(plateau,prochainCoup,1)}
      \instruction{retournerPion(joueurCourant)}
                  {
                    \sialors{plateauBloquePourUneCouleur(plateau,obtenirCouleurSuperieure(joueurCourant))}
                            {
                              \instruction{retournerPion(joueurCourant)}
                              \instruction{sortie(plateau,prochainCoup,0)}
                            }
                  }
                  \instruction{etatPartie(plateau,couleurGagnant,etat)}
    }{non(etat = partieEnCours)}
  }
\end{algorithme}

\vspace{5mm}

\begin{algorithme}
  \small
  \fonction
      {coupEnFctJoueur}
      {obtenirCoupBlanc,obtenirCoupNoir : \textbf{ObtenirCoupEnFctDuJoueur}, couleur : \textbf{Couleur}, plateau : \textbf{Plateau}}
      {\textbf{Coup}}
      {coup : \textbf{Coup}}
      {
        \sialorssinon{couleur = NOIR}
                {
                  \affecter{coup}{obtenirCoupNoir(plateau, NOIR, PROFONDEUR)}
                }
                {
                  \affecter{coup}{obtenirCoupBlanc(plateau, BLANC, PROFONDEUR)}
                }
                \retourner{coup}
      }
\end{algorithme}
