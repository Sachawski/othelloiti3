\begin{algorithme}
  \begin{enregistrement}{Case}
    \champEnregistrement{estVide}{\booleen}
    \champEnregistrement{pion}{Pion}
  \end{enregistrement}
  
  \vspace*{5mm}
  
\end{algorithme}

\vspace*{5mm}

\begin{algorithme}
  \type{\textbf{Plateau}}{Tableau[8][8] de \textbf{Case}}
\end{algorithme}

\vspace*{5mm}

\begin{algorithme}
  \small
  \fonction
      {plateau}
      {}
      {\textbf{Plateau}}
      {plateau : \textbf{Plateau} , i,j:\naturel}
      {
        \pour{i}{1}{8}{}
	     {
               \pour{j}{1}{8}{}
	            {
                      \affecter{plateau[i][j].estVide}{Vrai}
                    }
	     }
	     \retourner {plateau}
      }
\end{algorithme}

\vspace*{5mm}	

\begin{algorithme}
  \small
  \fonction
      {initialiser}
      {}
      {\textbf{Plateau}}
      {plateau : \textbf{Plateau} , pion : \textbf{Pion}, blanc :\textbf{Couleur}}
      {
	\affecter{plateau}{plateau()}
	\affecter{blanc}{couleurBlanc()}
	\affecter{pion}{creerPion(blanc)}       
    	\instruction{poserPion(plateau,position(4,4),pion)}
    	\instruction{poserPion(plateau,position(5,5),pion)}
    	\instruction{Pion.retournerPion(pion)}
    	\instruction{poserPion(plateau,position(4,5),pion)}
    	\instruction{poserPion(plateau,position(5,4),pion)}
    	\retourner{p}
      }
\end{algorithme}

\vspace*{5mm}

\begin{algorithme}
  \small
  \procedureAvecPreconditions
      {poserPion}
      {\paramEntreeSortie {plateau : \textbf{Plateau}} , \paramEntree {pos : \textbf{Position} , pion : \textbf{Pion}}}
      {estCaseVide(pos,plateau)}
      {x,y : \textbf{1...8}}
      {\affecter{x}{obtenirX(pos)}
    	\affecter{y}{obtenirY(pos)}
    	\affecter{plateau[x][y].pion}{pion}
    	\affecter{plateau[x][y].estVide}{Faux}
      }
\end{algorithme}

\vspace*{5mm}

\begin{algorithme}
  \small
  \fonctionAvecPreconditions
      {obtenirPion}
      {pos : \textbf{Position} , plateau : \textbf{Plateau}}
      {\textbf{Pion}}
      {non estCaseVide(plateau,pos)}
      {x,y : \textbf{1...8}}
      {
        \affecter{x}{obtenirX(pos)}
    	\affecter{y}{obtenirY(pos)}
    	\retourner {plateau[x][y].pion}
      }
\end{algorithme}

\vspace*{5mm}

\begin{algorithme}
  \small
  \procedureAvecPreconditions
      {retournerPion}
      {\paramEntreeSortie{plateau : \textbf{Plateau}} , \paramEntree{pos : \textbf{Position}}}
      {non estCaseVide(plateau,pos)}
      {pion : \textbf{Pion}}
      {
    	\affecter{pion}{obtenirPion(pos , plateau)}
    	\instruction{Pion.retournerPion(pion)}
    	\instruction{enleverPion(plateau,pos)}
    	\instruction{poserPion(plateau,pos,pion)}
      }
\end{algorithme}

\vspace*{5mm}

\begin{algorithme}
  \small
  \fonction
      {estCaseVide}{plateau :\textbf{Plateau} , pos : \textbf{Position}}
      {\booleen}
      {x,y : \textbf{1...8}}
      {
        \affecter{x}{obtenirX(pos)}
    	\affecter{y}{obtenirY(pos)}
    	\retourner {plateau[x][y].estVide}
      }
\end{algorithme}

\vspace*{5mm}

\begin{algorithme}
  \small
  \procedureAvecPreconditions
      {enleverPion}
      {\paramEntreeSortie{plateau : \textbf{Plateau}} , \paramEntree{pos : \textbf{Position}}}
      {non estCaseVide(plateau,pos)}
      {x,y : \textbf{1...8}}
      {
        \affecter{x}{obtenirX(pos)}
    	\affecter{y}{obtenirY(pos)}
        \affecter{plateau[x][y].estVide}{Vrai}
      }
      
      
\end{algorithme}

