\begin{algorithme}
	\small
	\fonction
	{obtenirCoup}
	{unPlateau : \textbf{Plateau}, joueur : \textbf{Couleur}, profondeur : \naturel}
	{\textbf{Coup}
	\preconditions{non plateauComplet(unPlateau)}}
	{resultat : \textbf{Coup}, cps : \textbf{Coups}, score, meilleurScore : \entier, i : \naturel}
	{\affecter{cps}{obtenirCoupsPossibles(unPlateau,joueur)}
		\affecter{resultat}{iemeCoup(cps,1)}
		\affecter{meilleurScore}{scoreDUnCoup(unPlateau,joueur,joueur,resultat,profondeur, INF\_NEGATIF, INF\_POSITIF)}
		\pour{i}{2}{nbCoups(cps)}{}
		{
			\affecter{score}{scoreDUnCoup(unPlateau,joueur,joueur,iemeCoup(cps,i),profondeur, INF\_NEGATIF, INF\_POSITIF))}
			\sialors{score $>$ meilleurScore}
			{
				\affecter{resultat}{iemeCoup(cps,i)}
				\affecter{meilleurScore}{score}
			}
		}
		\retourner{resultat}
	}
	
	\vspace*{5mm}
	
	\small
	\fonction
	{obtenirCoupsPossibles}
	{unPlateau : \textbf{Plateau}, joueurRef : \textbf{Couleur}}
	{\textbf{Coups}}
	{resultat : \textbf{Coups}, i, j : \naturel, coupTemp : \textbf{Coup}}
	{	\affecter{resultat}{coups()}
		\pour{i}{1}{HAUTEUR\_PLATEAU}{}
		{
			\pour{j}{1}{LARGEUR\_PLATEAU}{}
			{
				\affecter{coupTemp}{coup(creerPion(joueurRef),position(j,i))}
				\sialors{coupLegal(unPlateau,coupTemp)}
				{
					\instruction{ajouterCoups(resultat,couptemp)}
				}
			}
			
		}
		\retourner{resultat}
	}
	
	\vspace*{5mm}
	
	\small
	\fonction
	{evaluer}
	{unPlateau : \textbf{Plateau}, joueurRef : \textbf{Couleur}}
	{\entier}
	{}
	{	\retourner{score(unPlateau,joueurRef) - score(unPlateau,couleurChangerCouleur(joueurRef))}}
	
	\vspace*{5mm}
	
	\small
	\fonction
	{scoreDUnCoup}
	{unPlateau : \textbf{Plateau}, joueurRef, joueurCourant : \textbf{Couleur}, unCoup : \textbf{Coup}, profondeur : \naturel, alpha, beta : \entier}
	{\entier}
	{}
	{	\instruction{jouer(unPlateau,unCoup)}
		\sialorssinon{plateauBloque(unPlateau) ou profondeur = 0}
		{
			\retourner{evaluer(unPlateau,joueurRef)}
		}
		{
			\sialors{non(plateauBloquePourUneCouleur(unPlateau,couleurChangerCouleur(joueurCourant)))}
			{
				\affecter{joueurCourant}{couleurChangerCouleur(joueurCourant)}
			}
			
			\retourner{alphaBeta(unPlateau,joueurRef,joueurCourant,profondeur-1,alpha,beta)}
		}
		
	}
	
	\vspace*{5mm}
	
	\small
	\fonction
	{alphaBeta}
	{unPlateau : \textbf{Plateau}, joueurRef, joueurCourant : \textbf{Couleur}, profondeur : \naturel, alpha, beta : \entier}
	{\entier}
	{coupsPossibles : \textbf{Coups}, resultat, score : \entier, i : \naturel}
	{	\affecter{coupsPossibles}{obtenirCoupsPossibles(unPlateau,joueurCourant)}
		\affecter{resultat}{scoreDUnCoup(unPlateau,joueurRef,joueurCourant,iemeCoup(coupsPossibles,1), profondeur,alpha,beta)}
		\pour{i}{2}{nbCoups(coupsPossibles)}{}
		{
			\affecter{score}{scoreDUnCoup(unPlateau,joueurRef,joueurCourant,iemeCoup(coupsPossibles,i), profondeur,alpha,beta)}
			\sialorssinon{joueurCourant = joueurRef}
			{
				\affecter{resultat}{max(resultat,score)}
				\sialors{resultat $\leq$ alpha}
				{
					\retourner{resultat}
				}
				\affecter{beta}{min(beta,resultat)}
			}
			{
				\affecter{resultat}{min(resultat,score)}
				\sialors{beta $\leq$ resultat}
				{
					\retourner{resultat}
				}
				\affecter{alpha}{max(alpha,resultat)}
			}
		
		}
	\retourner{resultat}
		
	}
	
	\vspace*{5mm}
	
	\small
	\fonction
	{compteurCouleur}
	{plateau : \textbf{Plateau}, couleur : \textbf{Couleur}}
	{\naturel}
	{compteur : \textbf{Naturel}, i,j : \textbf{NNN}, pos : \textbf{Position}}
	{\affecter{compteur}{0}
		\pour{i}{1}{LARGEUR\_PLATEAU}{}
		{\pour{j}{1}{HAUTEUR\_PLATEAU}{}
			{\affecter{pos}{position(i,j)}
				\sialors{non(estCaseVide(pos,plateau)) et (obtenirCouleurSuperieur(obtenirPion(pos,plateau)) = couleur)}
				{\affecter{compteur}{compteur+1}}
			}
		}
		\retourner{compteur}
	}
	
	\vspace*{5mm}
	
	\small
	\fonction
	{plateauBloquePourUneCouleur}
	{unPlateau : \textbf{Plateau}, laCouleur : \textbf{Couleur}}
	{\booleen}
	{lesCoups : \textbf{Coups}}
	{\affecter{lesCoups}{obtenirCoupsPossibles(unPlateau,laCouleur)}
		\retourner{lesCoups.nbCoups = 0}
		
	}
	
	\vspace*{5mm}
	
	\small
	\fonction
	{plateauBloque}
	{unPlateau : \textbf{Plateau}}
	{\booleen}
	{}
	{\retourner{plateauBloquePourUneCouleur(unPlateau,NOIR) et plateauBloquePourUneCouleur(unPlateau,BLANC)}
	}
	
	\vspace*{5mm}
	
	\small
	\fonction
	{score}
	{unPlateau : \textbf{Plateau}, joueur : \textbf{Couleur}}
	{\entier}
	{score, sommePonderation : \textbf{Entier}, i,j : \textbf{Naturel}, ponderation : \tableauDeuxDimensions{HAUTEUR\_PLATEAU}{LARGEUR\_PLATEAU}{de}{\entier}, tempPos : \textbf{Position}}
	{
		\affecter{poderation}{[
				[100, -20, 10, 5, 5, 10, -20, 100],
				
				[-20, -50, -2, -2, -2, -2, -50, -20],
				
				[10, -2, -1, -1, -1, -1, -2, 10],
				
				[5, -2, -1, -1, -1, -1, -2, 5],
				
				[5, -2, -1, -1, -1, -1, -2, 5],
				
				[10, -2, -1, -1, -1, -1, -2, 10],
				
				[-20, -50, -2, -2, -2, -2, -50, -20],
				
				[100, -20, 10, 5, 5, 10, -20, 100]
			]}
		
		
		\pour{i}{1}{LARGEUR\_PLATEAU}{}
		{
			\pour{j}{1}{HAUTEUR\_PLATEAU}{}
			{
				\affecter{tempPos}{position(i,j)}
			 	\sialors{non(estCaseVide(tempPos,unPlateau)) et (obtenirCouleurSuperieur(obtenirPion(tempPos,unPlateau)) = joueur)}
				{
					\affecter{sommePonderation}{sommePonderation + ponderation[j][i]}
				}
			}
		}
		\affecter{score}{compteurCouleur(unPlateau,joueur) + sommePonderation}
		\retourner{score}
	}
	
	
\end{algorithme}
